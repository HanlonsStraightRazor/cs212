
\documentclass[12pt]{exam}

\usepackage[margin=1in]{geometry}
\usepackage{amsmath,amssymb}
\usepackage{color}
\usepackage{fancyvrb}

\newcommand{\class}{CS 212}
\newcommand{\term}{Spring 2019}
%\newcommand{\examnum}{Assignment 9}
\newcommand{\examnum}{The last assignment}
\newcommand{\examdate}{5/7/2019}

\pagestyle{head}
\firstpageheader{}{}{}
\runningheader{\class}{\examnum\ - Page \thepage\ of \numpages}{\examdate}
\runningheadrule


\begin{document}

\noindent
\begin{tabular*}{\textwidth}{l @{\extracolsep{\fill}} r @{\extracolsep{6pt}} l}
\textbf{\class} & \textbf{DUE: Tuesday, May 14 at 11:59 PM} & \\
\textbf{\term} &&\\
\textbf{\examnum} &&\\
\textbf{\examdate} &\textbf{10 points}&\\
\end{tabular*}\\
\rule[2ex]{\textwidth}{2pt}

\paragraph{Introduction:} This assignment will give you practice with analysis of algorithms. \\
\\
This is an individual effort homework assignment. You must write up your solutions in \LaTeX. Use the {\tt a9.tex} template that I provide and be sure to replace each ``Put your answer for \makebox[.25in]{\hrulefill} here.'' with your answers but leave everything else alone. Your solutions must be written up in a clear, concise and rigorous manner.  \\
\\
\noindent When you are done, zip up your .TEX file and corresponding .PDF file. Upload your .ZIP file to the {\bf a9} dropbox on d2l. After you have uploaded the file, double-check to ensure your file was uploaded correctly. It is your responsibility to ensure your submission was done correctly.  Assignments that are not uploaded correctly are worth 0 points. \\
\\
\\
Consider the following ``{\tt search}'' algorithm, which returns the index of {\tt x} in {\tt a} if {\tt x} exists in {\tt a} and {\tt -1} otherwise:
\begin{Verbatim}[frame=single]
static int search(int[] a, int x) {
  for (int h = a.length; h > 0; h /= 2)
    for (int i = a.length - 1; i >= 0; i -= h)
      if (a[i] == x)
        return i;
  return -1;
}\end{Verbatim}

\noindent The idea is that, you can pass in an {\tt int} array {\tt a} to search, along with a target value {\tt x}, for which to search, and the {\tt search} algorithm will either return a valid index, at which {\tt x} occurs in {\tt a}, or {\tt -1}, if {\tt x} does not occur in {\tt a}. You could do a simple test with the following code:
\begin{verbatim}
public static void main(String args[]) {
   int[] a = { 3, 14, 1, 5, 9, 2, 6 };
   int x = 1, /* x = 99 */
       i = search(a, x);
   System.out.println(i);
}
\end{verbatim}



\noindent The following parts will take you through a running time analysis of {\tt search}. This is a good example of the type of problem you may see early on in an Algorithms class (e.g., CS 321). 







\clearpage


\begin{questions}


	\question[1] \label{q1} Give a good definition for $N$, i.e., the \emph{size} of the input to {\tt search}.

	\fbox{\begin{minipage}[t]{14.55cm} \color{red}
		\vspace{25pt}
				
		\begin{center}
			$N \equiv $ the length (size) of the array to search
		\end{center}
				
		\vspace{25pt}
		\end{minipage}
	}





		
\clearpage







	\question[6] In this question, you will do a worst-case running time analysis of {\tt search}. Your goal will be to derive tight bounds, in terms of $\Theta$, on the worst-case running time of {\tt search}. 
	
	\begin{parts}
	
		\part[1] \label{q2a} First things first, give a clear yet concise description of a worst-case input to {\tt search}. Don't forget that this input must be of size $N$. 

		\fbox{\begin{minipage}[t]{14.55cm} \color{red}
			\vspace{25pt}
				
			\begin{center}
				The worst-case input to {\tt search} is an array of size $N$ which does not contain the element $x$.
			\end{center}
				
			\vspace{25pt}
			\end{minipage}
		}
		
		
		
		
		
		
\clearpage





	
		\part[2] \label{q2b} Give an asymptotic lower bound, using $\Omega$ notation, for the worst-case running time of {\tt search} that holds for \emph{at least one} input of size $N$. Be sure to specify exactly what the input is and then justify why your lower bound holds for it. Hint: see your answer to question~2\ref{q2a}. \\
		\\
		A bad asymptotic lower bound would be: $\Omega(1)$. The reason this is a bad answer, is because, for \emph{any} input, {\tt search} (or pretty much \emph{any} algorithm) must execute at least one simple operation. 

		\fbox{\begin{minipage}[t]{14.55cm} \color{red}
			\vspace{25pt}
				
			\begin{center}
				First, let's derive an expression that approximates the run time of {\tt search}. We can see that the outer for loop runs in approximately log${}_{2}N$ time, because each time the loop runs, it halves its control variable, which is initially set to be the size of the input $N$, and is allowed to go to a minimum of 1. The inner for loop runs in $N$ time because its control variable is initially set at $N-1$, but it is allowed to go to zero, unlike the outer for loop. For all intents and purposes, $h$ is a constant with respect to the inner for loop, therefore a constant amount is being subtracted from $i$ as it searches through the array. This makes its run time $N$. Since this is the worst case input, it will always do the if statement, which is a constant amount of work (call it 1), and the return inside of that will never execute. This means that the return at the end will execute, which is a constant time operation. This gives us the equation: $T(N) = 1 \cdot N\text{log}_{2}N + 1$. We can now derive constants $c$ and $N_{0}$ such that $1 \cdot N\text{log}_{2}N + 1 = \Omega(N\text{log}N)$:
				\begin{align*}
					1 \cdot N\text{log}_{2}N + 1 &\geq 1 \cdot N\text{log}_{2}N + 1 \\
					N\text{log}_{2}N + 1 &\geq N\text{log}_{2}N
				\end{align*}
				Now if we take $c = 1$ and $N_{0} = 2$, we get $3 \geq 2$ for the base case. We know this is true for all values of $N$ within the domain of each function, because the left hand side is always 1 greater than the right hand side. Therefore, our asymptotic lower bound is $\Omega(N\text{log}N)$.
			\end{center}
				
			\vspace{25pt}
			\end{minipage}
		}
	
	
	
	
	
\clearpage





	
		\part[2] \label{q2c} Give an asymptotic upper bound, using $O$ notation, for the worst-case running time of {\tt search} that holds for \emph{all} inputs of size $N$ and that matches the lower bound you gave in the previous part. Justify why your upper bound holds for all inputs. \\
		\\
		Here, $O\left(N^2\right)$ would be a valid answer, but it's probably too big of an upper-bound. It's valid, because, for any input, the outer loop does at most $N$ iterations, and, for each iteration of the outer loop, the inner loop does at most $N$ iterations, and in each iteration of the inner loop, $O(1)$ simple operations are executed. So, in total, {\tt search}, for any input, executes at most $O\left(N^2\right)$ simple operations. \\
		\\
		So, basically, try to figure out a \emph{lower} upper bound that is valid for all inputs of size $N$ that you could pass to {\tt search}. 

		\fbox{\begin{minipage}[t]{14.55cm} \color{red}
			\vspace{25pt}
				
			\begin{center}
				Now we'll just take the equation from above and use it to find $c$ and $N_{0}$ such that $1 \cdot N\text{log}_{2}N + 1 = O(N\text{log}N)$:
				\begin{align*}
					1 \cdot N\text{log}_{2}N + 1 &\leq 1 \cdot N\text{log}_{2}N + 1 \\
					N\text{log}_{2}N + 1 &\leq N\text{log}_{2}N + 1 \\
					&\leq N\text{log}_{2}N + N\text{log}_{2}N \\
					&\leq 2N\text{log}_{2}N
				\end{align*}
				Now let's let $c = 2$ and $N_{0} = 2$. This gives us $3 \leq 4$. We also know that $N\text{log}_{2}N + 1 \leq 2N\text{log}_{2}N$ for $N \geq N_{0}$ because $N\text{log}_{2}N \geq 1$ for $N \geq N_{0}$. Therefore, our asymptotic upper bound is $O(N\text{log}_{2}N)$.
			\end{center}
				
			\vspace{25pt}
			\end{minipage}
		}
	
	
	
	
	
	
	
	
\clearpage






\part[1] \label{q2d} Since the upper and lower bounds you gave in 2\ref{q2b} and 2\ref{q2c}, respectively, match (or, at least they're supposed to match), then what can you conclude about the asymptotic running time of {\tt search}?

\fbox{\begin{minipage}[t]{14.55cm} \color{red}
		\vspace{25pt}
		
		\begin{center}
			Since the asymptotic upper and lower bounds are the same, that means we can assign the asymptotic tight bound of $T(N) = \Theta(N\text{log}_{2}N)$ to our algorithm.
		\end{center}
		
		\vspace{25pt}
	\end{minipage}
}
	
	
	
	
	\end{parts}
	





\clearpage
	
	
	
	
	\question[3] In this question, you will do a best-case running time analysis of {\tt search}. Here, we will define the \emph{best-case} input to be the input of size $N$ that minimizes $T(N)$, where $T(N) = $ the maximum number of simple operations performed by the algorithm for some input of size $N$.
		
	\begin{parts}
	
	
		\part[1] \label{q3a} First things first, give a clear yet concise description of a best-case input to {\tt search}. Don't forget that this input must be of size $N$. 

		\fbox{\begin{minipage}[t]{14.55cm} \color{red}
			\vspace{25pt}
				
			\begin{center}
				The best case input to {\tt search} is an array of size $N$ where the element $x$ is at the end of the array (index $N - 1$). This is because {\tt search} always looks at the last index of the array for element $x$ first.
			\end{center}
				
			\vspace{25pt}
			\end{minipage}
		}
	
	
	
	
	
	
	
\clearpage






		\part[1] \label{q3b} Give an asymptotic upper bound, using $O$ notation, for the running time of {\tt search} that holds for \emph{some} input of size $N$. Be sure to specify exactly what the input is and then justify why your upper bound holds for it. Hint: see your answer to question~3\ref{q3a}. 

		
		\fbox{\begin{minipage}[t]{14.55cm} \color{red}
			\vspace{25pt}
				
			\begin{center}
				Given the best case input, the outer for loop does a constant amount of work initializing the control variable, the inner for loop does a constant amount of work initializing the control variable, the if statement inside the inner for loop does a constant amount of work checking to see if the last element in the array is $x$, (which is true for the best case input), and finally, the return statement inside of the if statement does a constant amount of work when it returns the value $i$. Therefore, since {\tt search} performs some constant amount of work given the best case input, $T(N) = O(1)$.
			\end{center}
				
			\vspace{25pt}
			\end{minipage}
		}
		
	
	
	
	
	
\clearpage	





		
		\part[1] \label{q3c} Finally, give an asymptotic lower bound, using $\Omega$ notation, for the running time of {\tt search} that holds for \emph{all} inputs of size $N$. Justify why your lower bound holds for all inputs. 
		
		\fbox{\begin{minipage}[t]{14.55cm} \color{red}
			\vspace{25pt}
				
			\begin{center}
				Since all algorithms run in at least constant time, and our asymptotic upper bound is $O(1)$, the asymptotic lower bound of $T(N)$ is $\Omega(1)$ because our asymptotic lower bound cannot be greater than our asymptotic upper bound.
			\end{center}
				
			\vspace{25pt}
			\end{minipage}
		}
		
		
		

	
	
	\end{parts}
		
	
\end{questions}
		


\end{document}
